\pagenumbering{roman} 
\setcounter{page}{1} 
\pagestyle{plain}

%%%%%%%%%%%%%%%%
%%% CREDITOS %%%
%%%%%%%%%%%%%%%%
% \chapter*{Créditos/Copyright}

% Una página con la especificación de créditos/copyright para el proyecto (ya sea aplicación por un lado y documentación por el otro, o unificadamente), así como la del uso de marcas, productos o servicios de terceros (incluidos códigos fuente). Si una persona diferente al autor colaboró en el proyecto, tiene que quedar explicitada su identidad y qué hizo.

% A continuación se ejemplifica el caso más habitual, aunque se puede modificar por cualquier otra alternativa:

\vspace{1cm}

\begin{figure}[ht]
    \centering
	\includegraphics[scale=1]{images/license.png}
\end{figure}

Esta obra está sujeta a una licencia de Reconocimiento -  NoComercial - SinObraDerivada

\href{https://creativecommons.org/licenses/by-nc-nd/3.0/es/}{3.0 España de CreativeCommons}.

%%%%%%%%%%%%%
%%% FICHA %%%
%%%%%%%%%%%%%
\chapter*{FICHA DEL TRABAJO FINAL}

\begin{table}[ht]
	\centering{}
	\renewcommand{\arraystretch}{2}
	\begin{tabular}{r | p{10cm}}
		\hline
		Título del trabajo: & IA Generativa para la recuperación 
		de información de convocatorias de ayudas a empresas\\
		\hline
        Nombre del autor: & José Luis Rodríguez Andreu\\
		\hline
        Nombre del colaborador/a docente: & Diego Calvo Barreno\\
		\hline
        Nombre del PRA: & Josep Anton Mir Tutusaus\\
		\hline
        Fecha de entrega (mm/aaaa): & 06/2025\\
		\hline
        Titulación o programa: & Máster Universitario en Ciencia de Datos\\
		\hline
        Área del Trabajo Final: & Trabajo Fin de Máster\\
		\hline
        Idioma del trabajo: & Español\\
		\hline
        Palabras clave & LLM, RAG, AI\\
		\hline
	\end{tabular}
\end{table}

%%%%%%%%%%%%%%%%%%%
%%% DEDICATORIA %%%
%%%%%%%%%%%%%%%%%%%
\chapter*{Dedicatoria/Cita}

Breves palabras de dedicatoria y/o una cita.
%A Miguel, que lleva el nombre de un poeta valiente, y a Carmen, 

%%%%%%%%%%%%%%%%%%%
%%% Agradecimientos %%%
%%%%%%%%%%%%%%%%%%%
\chapter*{Agradecimientos}

Si se considera oportuno, mencionar a las personas, empresas o instituciones que hayan contribuido en la realización de este proyecto.

%%%%%%%%%%%%%%%%
%%% Abstract  %%%
%%%%%%%%%%%%%%%%
\chapter*{Abstract}
\addcontentsline{toc}{chapter}{Abstract}

\onehalfspacing

In recent years, it has become increasingly difficult to find calls for financial aid from governmental institutions focused on companies and organizations. This growing need makes it essential to have systems that optimize the identification of calls for financial aid. 

Currently, the lack of automated tools capable of interpreting and synthesizing available information hinders efficient access to these resources, forcing organizations to conduct manual searches that consume both time and effort.

This work presents the development of an Artificial Intelligence (AI)-based tool for extracting and retrieving information from economic aid calls.

By leveraging advanced Natural Language Processing (NLP) techniques and Generative AI, the solution can analyze, structure, and filter information automatically, providing relevant results based on the specific characteristics of each entity.

The main objective of the tool is to process and transform scattered aid calls into a structured dataset, facilitating their consultation and retrieval. This organized structure will allow companies to quickly and efficiently access the most relevant information, enhancing strategic decision-making.

In this way, the project addresses the challenge of filtering and synthesizing large volumes of unstructured and dispersed data from various platforms, streamlining the search process and improving access to funding opportunities.

\vspace{1.5cm}

\textbf{Keywords}: LLMs, IA, RAG

%%%%%%%%%%%%%%%%
%%% RESUMEN  %%%
%%%%%%%%%%%%%%%%
\chapter*{Resumen}
\addcontentsline{toc}{chapter}{Resumen}

\onehalfspacing

En los últimos años cada vez es mas complicado encontrar convocatorias de ayuda economica por parte de instituciones gubernamentales enfocadas a empresas y entidades. Esta creciente necesidad hace imprescindible contar con sistemas que optimicen la identificación de convocatorias de ayudas económicas. 

Actualmente, la ausencia de herramientas automatizadas que interpreten y sinteticen la información disponible dificulta el acceso eficiente a estos recursos, obligando a las organizaciones a realizar búsquedas manuales que consumen tiempo y recursos.
Este trabajo presenta el desarrollo de una herramienta basada en Inteligencia Artificial (IA) para la extracción y recuperación de información de convocatorias de ayudas económicas.

Utilizando técnicas avanzadas de Procesamiento de Lenguaje Natural (NLP) e IA Generativa, la solución permite analizar, estructurar y filtrar la información de manera automatizada, proporcionando resultados relevantes en función de las características específicas de cada entidad.

El objetivo principal de la herramienta es procesar y convertir las convocatorias de ayudas dispersas en un conjunto de datos estructurados, lo que facilita su consulta y recuperación. Esta estructura organizada permitirá a las empresas acceder de forma rápida y eficiente a la información más relevante, mejorando la toma de decisiones estratégicas. 

De este modo, se aborda el desafío de filtrar y sintetizar grandes volúmenes de datos no estructurados y dispersos de diversas plataformas, simplificando el proceso de búsqueda y optimizando el acceso a oportunidades de financiamiento.

\vspace{1.5cm}

\textbf{Palabras clave}: LLMs, IA, RAG