\documentclass[12pt,a4paper,twoside]{book}
\usepackage{graphicx}
\usepackage{setspace} % espaciado doble para texto, simple para pies de página, subtítulos, etc.
\usepackage{natbib} % sustituto de 'hypernat' que funciona en Windows.
\usepackage[spanish]{babel}
\usepackage[utf8]{inputenc}
\usepackage{color}
\usepackage{hhline} % estilos extendidos para tablas
\usepackage{multirow}
\usepackage{subfigure}
\usepackage{acronym}
\usepackage{hyperref}
\usepackage{amsmath,amssymb}
\usepackage{fancyhdr}
\usepackage{epsfig, amsmath}
\usepackage{algorithm}
\usepackage{algorithmic}

% configuraciones generales
\hypersetup{
linktocpage=true,
colorlinks=true,
linkcolor=blue,
citecolor=blue,
}
\definecolor{Hgray}{gray}{0.6}

\newenvironment{definition}[1][Definición]{\begin{trivlist}
\item[\hskip \labelsep {\bfseries #1}]}{\end{trivlist}}

\setlength{\topmargin}{0cm}
\setlength{\textheight}{23cm}
\setlength{\textwidth}{17cm}
\setlength{\oddsidemargin}{0cm}
\setlength{\evensidemargin}{0cm}
\setlength{\headheight}{1cm}

% indica que las 'sub-sub-secciones' están numeradas y aparecen en el índice
\setcounter{secnumdepth}{3}
\setcounter{tocdepth}{2}

% configuraciones para código
\renewcommand{\algorithmicrequire}{\textbf{Entrada:}}
\renewcommand{\algorithmicensure}{\textbf{Salida:}}

%%%%%%%%%%%%
% DOCUMENTO %
%%%%%%%%%%%%
\begin{document}

\setcounter{section}{0} % Restablece el contador de sección a 0 al inicio del documento
\renewcommand{\thesection}{\arabic{section}} % Cambia el esquema de numeración de sección

% portada
\newpage
\thispagestyle{empty}

\baselineskip 2em

%\vspace*{1cm}

\centerline{\includegraphics[width=0.6\textwidth]{images/UOC-logo}}
\begin{center}
\textsc{Universitat Oberta de Catalunya (UOC) \\
Máster Universitario en Ciencia de Datos (\textit{Data Science})\\}

%\centerline {\pic{UOC}{4cm}}

\vspace*{1.5cm}

\textsc{\Large TRABAJO FINAL DE MÁSTER}

\vspace*{0.5cm}

\textsc{\large Área: Natural Language Processing and Visual Analytics\\}
\textsc{\large Data Mining, Graphs and Natural Language Processing}

%\textbf{\Huge VirtualTechLab Model: }

\vspace*{2.0cm}

\textbf{\Large IA Generativa para la recuperación de información de convocatorias de ayudas a empresas}

%\textbf{\large xxx subtítulo (en caso de existir) xxx}

\vspace{2.5cm}
\baselineskip 1em

\baselineskip 2em
-----------------------------------------------------------------------------\\
Autor:      José Luis Rodríguez Andreu\\
Tutor:      Diego Calvo Barreno\\
Profesor:   Josep Anton Mir Tutusaus\\
-----------------------------------------------------------------------------\\
\vspace*{1.5cm}
Barcelona, \today

\end{center}

\newpage
\pagestyle{empty}
\hfill

\newpage
% resumen
\pagenumbering{roman} 
\setcounter{page}{1} 
\pagestyle{plain}

%%%%%%%%%%%%%%%%
%%% CREDITOS %%%
%%%%%%%%%%%%%%%%
% \chapter*{Créditos/Copyright}

% Una página con la especificación de créditos/copyright para el proyecto (ya sea aplicación por un lado y documentación por el otro, o unificadamente), así como la del uso de marcas, productos o servicios de terceros (incluidos códigos fuente). Si una persona diferente al autor colaboró en el proyecto, tiene que quedar explicitada su identidad y qué hizo.

% A continuación se ejemplifica el caso más habitual, aunque se puede modificar por cualquier otra alternativa:

\vspace{1cm}

\begin{figure}[ht]
    \centering
	\includegraphics[scale=1]{images/license.png}
\end{figure}

Esta obra está sujeta a una licencia de Reconocimiento -  NoComercial - SinObraDerivada

\href{https://creativecommons.org/licenses/by-nc-nd/3.0/es/}{3.0 España de CreativeCommons}.

%%%%%%%%%%%%%
%%% FICHA %%%
%%%%%%%%%%%%%
\chapter*{FICHA DEL TRABAJO FINAL}

\begin{table}[ht]
	\centering{}
	\renewcommand{\arraystretch}{2}
	\begin{tabular}{r | p{10cm}}
		\hline
		Título del trabajo: & IA Generativa para la recuperación 
		de información de convocatorias de ayudas a empresas\\
		\hline
        Nombre del autor: & José Luis Rodríguez Andreu\\
		\hline
        Nombre del colaborador/a docente: & Diego Calvo Barreno\\
		\hline
        Nombre del PRA: & Josep Anton Mir Tutusaus\\
		\hline
        Fecha de entrega (mm/aaaa): & 06/2025\\
		\hline
        Titulación o programa: & Máster Universitario en Ciencia de Datos\\
		\hline
        Área del Trabajo Final: & Trabajo Fin de Máster\\
		\hline
        Idioma del trabajo: & Español\\
		\hline
        Palabras clave & LLM, RAG, AI\\
		\hline
	\end{tabular}
\end{table}

%%%%%%%%%%%%%%%%%%%
%%% DEDICATORIA %%%
%%%%%%%%%%%%%%%%%%%
\chapter*{Dedicatoria/Cita}

Breves palabras de dedicatoria y/o una cita.
%A Miguel, que lleva el nombre de un poeta valiente, y a Carmen, 

%%%%%%%%%%%%%%%%%%%
%%% Agradecimientos %%%
%%%%%%%%%%%%%%%%%%%
\chapter*{Agradecimientos}

Si se considera oportuno, mencionar a las personas, empresas o instituciones que hayan contribuido en la realización de este proyecto.

%%%%%%%%%%%%%%%%
%%% Abstract  %%%
%%%%%%%%%%%%%%%%
\chapter*{Abstract}
\addcontentsline{toc}{chapter}{Abstract}

\onehalfspacing

In recent years, it has become increasingly difficult to find calls for financial aid from governmental institutions focused on companies and organizations. This growing need makes it essential to have systems that optimize the identification of calls for financial aid. 

Currently, the lack of automated tools capable of interpreting and synthesizing available information hinders efficient access to these resources, forcing organizations to conduct manual searches that consume both time and effort.

This work presents the development of an Artificial Intelligence (AI)-based tool for extracting and retrieving information from economic aid calls.

By leveraging advanced Natural Language Processing (NLP) techniques and Generative AI, the solution can analyze, structure, and filter information automatically, providing relevant results based on the specific characteristics of each entity.

The main objective of the tool is to process and transform scattered aid calls into a structured dataset, facilitating their consultation and retrieval. This organized structure will allow companies to quickly and efficiently access the most relevant information, enhancing strategic decision-making.

In this way, the project addresses the challenge of filtering and synthesizing large volumes of unstructured and dispersed data from various platforms, streamlining the search process and improving access to funding opportunities.

\vspace{1.5cm}

\textbf{Keywords}: LLMs, IA, RAG

%%%%%%%%%%%%%%%%
%%% RESUMEN  %%%
%%%%%%%%%%%%%%%%
\chapter*{Resumen}
\addcontentsline{toc}{chapter}{Resumen}

\onehalfspacing

En los últimos años cada vez es mas complicado encontrar convocatorias de ayuda economica por parte de instituciones gubernamentales enfocadas a empresas y entidades. Esta creciente necesidad hace imprescindible contar con sistemas que optimicen la identificación de convocatorias de ayudas económicas. 

Actualmente, la ausencia de herramientas automatizadas que interpreten y sinteticen la información disponible dificulta el acceso eficiente a estos recursos, obligando a las organizaciones a realizar búsquedas manuales que consumen tiempo y recursos.
Este trabajo presenta el desarrollo de una herramienta basada en Inteligencia Artificial (IA) para la extracción y recuperación de información de convocatorias de ayudas económicas.

Utilizando técnicas avanzadas de Procesamiento de Lenguaje Natural (NLP) e IA Generativa, la solución permite analizar, estructurar y filtrar la información de manera automatizada, proporcionando resultados relevantes en función de las características específicas de cada entidad.

El objetivo principal de la herramienta es procesar y convertir las convocatorias de ayudas dispersas en un conjunto de datos estructurados, lo que facilita su consulta y recuperación. Esta estructura organizada permitirá a las empresas acceder de forma rápida y eficiente a la información más relevante, mejorando la toma de decisiones estratégicas. 

De este modo, se aborda el desafío de filtrar y sintetizar grandes volúmenes de datos no estructurados y dispersos de diversas plataformas, simplificando el proceso de búsqueda y optimizando el acceso a oportunidades de financiamiento.

\vspace{1.5cm}

\textbf{Palabras clave}: LLMs, IA, RAG
\newpage

\pagestyle{fancy}
\renewcommand{\chaptermark}[1]{ \markboth{#1}{}}
\renewcommand{\sectionmark}[1]{\markright{ \thesection.\ #1}}
\lhead[\fancyplain{}{\bfseries\thepage}]{\fancyplain{}{\bfseries\rightmark}}
\rhead[\fancyplain{}{\bfseries\leftmark}]{\fancyplain{}{\bfseries\thepage}}
\cfoot{}

% tabla de contenidos
\cleardoublepage
\phantomsection
\addcontentsline{toc}{chapter}{Índice}
\tableofcontents
% lista de figuras
\cleardoublepage
\phantomsection
\addcontentsline{toc}{chapter}{Lista de Figuras}
\listoffigures
% lista de tablas
\cleardoublepage
\phantomsection
\addcontentsline{toc}{chapter}{Lista de Tablas}
\listoftables

\thispagestyle{empty}

\pagenumbering{arabic}

\pagestyle{fancy}
\renewcommand{\chaptermark}[1]{ \markboth{#1}{}}
\renewcommand{\sectionmark}[1]{\markright{ \thesection.\ #1}}
\lhead[\fancyplain{}{\bfseries\thepage}]{\fancyplain{}{\bfseries\rightmark}}
\rhead[\fancyplain{}{\bfseries\leftmark}]{\fancyplain{}{\bfseries\thepage}}
\cfoot{}

\onehalfspacing

% capítulos del documento
\section{Introducción}

Esta plantilla pretende ser una guía para los estudiantes. Esta plantilla se puede adaptar a las necesidades específicas de cada proyecto si el supervisor del proyecto está de acuerdo con los cambios.

\begin{figure}[h]
\centering
\includegraphics[width=0.5\textwidth]{./figs/image1.png}
\caption{Ejemplo de figura. Será indexada en la “Lista de Figuras”.}
\label{fig:figura_ejemplo}
\end{figure}

\subsection{Contexto y motivación}

Punto de partida del proyecto (¿Cuál es el problema que necesita ser resuelto? ¿Por qué es un tema relevante? ¿Cómo se está resolviendo el problema actualmente?) y descripción de la contribución (¿Qué resultado se espera obtener?)

\subsection{Objetivos}

Listado de los objetivos a alcanzar en este proyecto.

\subsection{Sostenibilidad, diversidad y desafíos ético/sociales}

Esta sección debe evaluar el impacto positivo/negativo del proyecto en las siguientes dimensiones. No es necesario alcanzar un impacto positivo en todas las dimensiones, pero es necesario considerar y discutir si existe un impacto desde el inicio del proyecto.

\begin{description}
    \item[Sostenibilidad] En el desarrollo del proyecto o durante todo su ciclo de vida (por ejemplo, despliegue, retiro), ¿tiene el resultado de este proyecto un impacto en la sostenibilidad y/o huella ecológica (consumo/ahorro de energía/recursos, desperdicio, contaminación, agotamiento de materias primas)? ¿Está afectado por leyes o regulaciones sobre este asunto? Considerando otra perspectiva, ¿afecta a alguno de los Objetivos de Desarrollo Sostenible (ODS) relacionados con estas dimensiones? Si no tiene ningún impacto, ya sea positivo o negativo, debe explicar cómo llegó a esta conclusión y justificar su respuesta.
    \item[Comportamiento ético y responsabilidad social] ¿Es el resultado del proyecto demasiado técnico para tener algún impacto positivo/negativo en aspectos éticos/sociales? ¿Tiene un impacto en leyes/regulaciones (datos, privacidad, trabajo, propiedad intelectual, seguridad personal, …)? ¿Se adhiere a los principios deontológicos de la profesión? ¿Pone en peligro/mejora/empeora algún puesto de trabajo? Si no tiene ningún impacto, ya sea positivo o negativo, debe explicar cómo llegó a esta conclusión y justificar su respuesta.
    \item[Diversidad, género y derechos humanos] ¿Es el resultado de este proyecto tan técnico que no tiene impacto positivo/negativo en términos de género, diversidad o derechos humanos? ¿Y en leyes/regulaciones? ¿Y en términos de accesibilidad, discapacidad, ergonomía y/o seguridad de la información? Si no tiene ningún impacto, ya sea positivo o negativo, debe explicar cómo llegó a esta conclusión y justificar su respuesta.
\end{description}

\subsection{Enfoque y metodología}

Describa las estrategias potenciales para desarrollar este proyecto y explique la estrategia seleccionada. Discuta por qué la estrategia seleccionada es la más adecuada para alcanzar los objetivos del proyecto.

\subsection{Planificación}

Descripción de los recursos necesarios para desarrollar el proyecto, las tareas a realizar y una planificación temporal de cada tarea utilizando un diagrama de Gantt o equivalente. Esta planificación debe definir los hitos que se completarán en cada Prueba de Evaluación Continua (PEC).

\subsection{Resumen de los productos del proyecto}

No es necesario describir cada producto en detalle: esto se hará en los capítulos restantes del proyecto.

\subsection{Breve descripción de los demás capítulos del informe}

Breve descripción de los contenidos de cada capítulo y su relación con el resto del proyecto.

\section{Métodos y recursos}

En estas secciones, es necesario describir:

\begin{itemize}
    \item Los aspectos más relevantes del diseño y desarrollo del proyecto.
    \item La metodología utilizada en el proceso de desarrollo, describiendo las alternativas posibles, las decisiones que se han tomado y los criterios utilizados para tomar estas decisiones.
    \item Una descripción de los productos que se han creado.
\end{itemize}

La estructura de estas secciones puede cambiar según el tipo de proyecto que se esté desarrollando.

\section{Resultados}

Describa los resultados obtenidos utilizando la metodología descrita anteriormente.

\section{Conclusiones y trabajo futuro}

Esta sección debe incluir lo siguiente:

\begin{itemize}
    \item Una descripción de las conclusiones del trabajo.
    \item Una evaluación crítica del grado de logro de los objetivos iniciales.
    \item Una evaluación crítica de la planificación y metodología utilizadas en el proyecto.
    \item Considerando los desafíos de sostenibilidad, diversidad y ético-sociales vinculados al proyecto.
    \item Una discusión de temas para trabajo futuro potencial que no se hayan explorado en este proyecto.
\end{itemize}

\section{Glosario}

Definición de los términos y acrónimos más relevantes utilizados en este informe.

% bibliografía
\addcontentsline{toc}{chapter}{Bibliografía}
\bibliographystyle{plain}
\bibliography{referencias}

\section{Anexos}

Lista de secciones que son demasiado largas para ser incluidas en el cuerpo del informe y que son autocontenidas.

\end{document}
