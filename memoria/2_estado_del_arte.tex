\chapter{Estado del Arte}
\label{chapter:Estado del Arte}


%%% SECTION
\section{Introducción}
El objetivo de este capítulo es realizar un análisis de los diferentes avances, desarrollos y tecnologías disponibles en el ámbito de la solución planteada. 

Este análisis tiene como objetivo identificar enfoques y metodologías en distintas áreas, como la extracción de información a partir de fuentes web, el análisis y procesado de texto y el uso de técnicas de Inteligencia Artificial aplicadas al Procesamiento de Lenguaje Natural.

De esta forma se puede establecer un contexto para el problema a resolver y justificar la elección de las tecnologías y metodologías a utilizar en el desarrollo de la solución propuesta.

\section{Problemática a resolver}

La búsqueda de ayudas y subvenciones es una tarea que la mayoría de las empresas, sobre todo las que tienen menos recursos, realizan en su día a día.
Para ello, existen diferentes plataformas de ayudas a empresas, algunas nacionales y otras de carácter internacional. 
Sin embargo, esta tarea puede resultar complicada y tediosa, ya que implica una búsqueda constante de nuevas posibilidades de financiacióna través de distintas fuentes.
Además, la información sobre estas convocatorias suele estar distribuidas en diferentes fuentes, desde las propias plataformas a documentación oficial del estado.
Esto supone que a la hora de realizar una búsqueda de posibles convocatorias de financiación, se acabe con un conjunto de fuentes con diferentes estructuras y formatos.

En la mayoría de los casos, las convocatorias suelen tener asociados diferentes documentos, en su mayoría en formato PDF, los cuales pueden ser extensos, y además usan un lenguaje técnico, típico de este tipo de documentos, que dificulta su comprensión.
Esto al final supone una complicación por parte de las empresas a la hora de acceder a información clave de las convocatorias de forma mas rápida, como requisitos, plazos, presupuesto o condiciones de participación.
Estos problemas de accesibilidad y estandarización de las convocatorias de ayudas suponen una barrera de acceso importante, que reduce las oportunidades de acceso a financiación para algunas empresas, y suponen una inversión en tiempo y esfuerzo en la tarea de búsqueda y filtrado por parte de éstas

El desarrollo planteado en este proyecto pretende ser una solución a esta problemática, proporcionando una herramienta que sea capaz de identificar y extraer la documentación de las convocatorias, y aplicar técnicas de Inteligencia Artificial para extraer la información clave y dotarla de una estructura mas estandarizada, así como permitir la consulta de esta información de forma sencilla a partir de un agente conversacional.


% \section{Soluciones clásicas}


% \subsection{Plataformas de convocatorias}

% \subsection{Web Scraping}

% \subsection{Procesamiento de Lenguaje Natural}

% \section{Inteligencia Artificial Generativa}

% \subsection{Grandes Modelos del Lenguaje (LLMs)}

% \subsection{Prompt Engineering}

% \subsection{Retrieval Augmented Generation (RAG)}

% \subsection{Agentes Inteligentes}

% \section{Tecnologías y métodos actuales}

% \section{conclusiones}